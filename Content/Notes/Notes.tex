\documentclass[10pt, oneside, reqno]{amsart}
\usepackage{geometry, setspace, graphicx, enumerate}
\onehalfspacing

% AMS Theorems
\theoremstyle{plain}% default
\newtheorem{thm}{Theorem}[section]
\newtheorem{lem}[thm]{Lemma}
\newtheorem{prop}[thm]{Proposition}
\newtheorem*{cor}{Corollary}

\theoremstyle{definition}
\newtheorem{definition}[thm]{Definition}
\newtheorem{conj}[thm]{Conjecture}
\newtheorem{alg}[thm]{Algorithm}
\newtheorem{exmp}[thm]{Example}
\theoremstyle{remark}
\newtheorem*{rem}{Remark}
\newtheorem*{note}{Note}
\newtheorem{case}{Case}

\newcommand{\expc}[1]{\mathbb{E}\left[#1\right]}
\newcommand{\var}[1]{\text{Var}\left(#1\right)}
\newcommand{\cov}[1]{\text{Cov}\left(#1\right)}

\newcommand{\R}{\mathbb{R}}
\newcommand{\C}{\mathbb{C}}
\newcommand{\K}{\mathbb{K}}

\usepackage{amsmath}	%Improves output of document containing mathematical formulas
\usepackage{amsthm}		%Enhanced version of \newtheorem command for defining theorem-like structures
\usepackage{amssymb}	%Extended symbol collection - additional binary relation symbols. Contains \amsfonts
\usepackage{amscd}
\usepackage{mathtools}	% Having piecemeal on the right hand side
\usepackage{mdframed}

\usepackage{xcolor}  % Allows the addition of colours to text
\usepackage{hyperref}
\hypersetup{
    linktoc=all,     %set to all if you want both sections and subsections linked
}

\usepackage{graphicx}	%Allows you to insert images
\graphicspath{ {./figs/} }	%Images are stored in the images folder in current directory'
\usepackage{float}
\usepackage{caption}	%Allows captions to be inserted with figures
\usepackage{listings}	%Writing code

\usepackage{algorithm} % Writing Algorithms
\usepackage[noend]{algpseudocode}


\title{DATA5441 - Networks and High-Dimensional Inference}                                % Document Title
\author{Influentials}
\date{}                                           % Activate to display a given date or no date


\begin{document}
\maketitle \tableofcontents \clearpage

\section{Introduction}

\subsection{The Influentials Hypothesis}
\texttt{Influentials} are a minority of individuals who influence an exceptional 
number of their peers. 

The \textbf{hypothesis} was that influentials were mediators between the source of innovation and the majority 
of society. The model, called the \texttt{two-step flow} of communication.

\subsection{Questions of Interest}
\begin{itemize}
    \item What does the two-step model say about influentials?
    \item How do influentials exert influence over the larger population?
    \item Are influentials responsible for the spread/diffusion of innovation?
\end{itemize}


\subsection{General Results}

\begin{itemize}
    \item Under certain (rare) conditions, influentials appear responsible for 
    initiating cascades of influence and are important.
    \item Under most conditions, most cascades are driven by easily influenced individuals
    influencing other easily influenced individuals.
\end{itemize}


\begin{itemize}
    \item Model: Two-way influence model with influentials acting as a middle-man.
    \item Simulations suggest that under certain conditions, influentials promote 
    cascading effects but these conditions are rare.
    \item Computer simulation models 
\end{itemize}


Simulations 

Threshold Model
\begin{itemize}
    \item Each individual $i$ in a population of $N$ influence $n_i$ random individuals.
    \item Early adopters are individuals who adopt an innovation when a single neighbour has innovated.
\end{itemize}




\end{document}



