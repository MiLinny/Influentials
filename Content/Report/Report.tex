\documentclass[10pt, oneside, reqno]{amsart}
\usepackage{geometry, setspace, graphicx, enumerate}
\onehalfspacing

% AMS Theorems
\theoremstyle{plain}% default
\newtheorem{thm}{Theorem}[section]
\newtheorem{lem}[thm]{Lemma}
\newtheorem{prop}[thm]{Proposition}
\newtheorem*{cor}{Corollary}

\theoremstyle{definition}
\newtheorem{definition}[thm]{Definition}
\newtheorem{conj}[thm]{Conjecture}
\newtheorem{alg}[thm]{Algorithm}
\newtheorem{exmp}[thm]{Example}
\theoremstyle{remark}
\newtheorem*{rem}{Remark}
\newtheorem*{note}{Note}
\newtheorem{case}{Case}

\newcommand{\expc}[1]{\mathbb{E}\left[#1\right]}
\newcommand{\var}[1]{\text{Var}\left(#1\right)}
\newcommand{\cov}[1]{\text{Cov}\left(#1\right)}

\newcommand{\R}{\mathbb{R}}
\newcommand{\C}{\mathbb{C}}
\newcommand{\K}{\mathbb{K}}

\makeatletter
\renewcommand\subsection{\@startsection{subsection}{2}%
  \z@{.5\linespacing\@plus.7\linespacing}{-.5em}%
  {\normalfont\scshape}}
\makeatother

\usepackage{amsmath}	%Improves output of document containing mathematical formulas
\usepackage{amsthm}		%Enhanced version of \newtheorem command for defining theorem-like structures
\usepackage{amssymb}	%Extended symbol collection - additional binary relation symbols. Contains \amsfonts
\usepackage{amscd}
\usepackage{mathtools}	% Having piecemeal on the right hand side
\usepackage{mdframed}

\usepackage{xcolor}  % Allows the addition of colours to text
\usepackage{hyperref}
\hypersetup{
    linktoc=all,     %set to all if you want both sections and subsections linked
}

\usepackage{graphicx}	%Allows you to insert images
\graphicspath{ {./figs/} }	%Images are stored in the images folder in current directory'
\usepackage{float}
\usepackage{caption}	%Allows captions to be inserted with figures
\usepackage{listings}	%Writing code

\usepackage{algorithm} % Writing Algorithms
\usepackage[noend]{algpseudocode}

\title{Influentials and the Spread of Innovations}                                % Document Title
\author{Michael Lin\\470414095}

\begin{document}
\vspace*{-1.5cm}
\begin{abstract}
    This report explores the \textit{influential hypothesis} - a minority of individuals, called 
    influentials, who can spread innovations to an exceptional number of peers are important to 
    the spread of ideas/innovations. We explore this idea by performing computer simulations 
    modelling the interpersonal influence process under various conditions. It was found that 
    the importance of influentials in the spread of influence depended on the degree distribution
    of the network which suggests a reexamination of the \textit{influential hypothesis}.
\end{abstract}
\maketitle 

\section{Introduction}

The spread of influence is a well studied phenomena in social science and diffusion research 
that looks to understand how influence is spread and what factors contribute to large cascades
of influence within a population. 
This phenomena can be found in many domains such as the circulation of news by the media, the 
adoption of new technologies, and the promotion of products by social media influencers.
One hypothesis on how innovations are spread is called the \textit{influential hypothesis}\cite{Influential},
which suggests that a minority of individuals called \textit{influentials} are able to spread 
ideas/innovations to an exceptional number of peers. 
However, this model does not explain the characteristics of influentials or precisely how 
responsible they are in the cascade of influence on the (non-influential) population.

In this report, we argue that it is unclear what characteristics underpin influentials and 
whether they can be attributed to the adoption of social changes, new technologies, cultural 
fads and other diffusion processes.
By performing a series of computer simulations of various network models, it was found that 
there are instances where influentials show greater responsibility in the spread of influences.
However, under other conditions it was found that influentials are only marginally more important
than the average individual. 
Although our models are simplifications of the complexities of reality, our results highlights
the importance of population interconnectedness in dictating the responsibility of influentials
in the cascade of influence.




% In today's 

% Influentials - a minority within a population who influence an exceptional number of their peers.



\section{Interpersonal Influence Model}

For our model of the spread of influence, we assume that every individual makes a binary decision
on an innovation $X$. Moreover, we assume that the innovation exhibits \textit{positive externalities}
meaning the probability of an individual choosing $B$ over $A$ increases with the number of 
people choosing $B$. While this model is not fully general, as it excludes negative externality 
behaviours and innovations with multiple decisions, it is a reasonable general case to consider. 
For example, in marketing and diffusion research, positive externalities arise in many areas of 
research such as \textit{network effects}, \textit{learning from others}, and conformity 
pressures \cite{Influential}.

\subsection{Threshold Model}
A simple model to capture the aforementioned behaviour is to represent each individual $i$ by a node,
which can be in one of two discrete states $\{ 0, 1\}$, with a \textit{threshold rule} $\phi_i$. 
An individual in state $1$ is said to be \texttt{influenced} by an innovation while an 
individual in state $0$ is uninfluenced by an innovation.
Defining $p_i$ to be the proportion of $i$'s neighbours who are influenced (in state $1$), then 
the probability of $i$ being influenced (in stat $1$) is given by
\[ P[\text{Node } i \text{ in state } 1] = 
\begin{cases}
    1 & p_i \geq \phi_i \\
    0 & p_i < \phi_i
\end{cases} \]
where $\phi_i \in [0,1]$ refers to the minimum proportion of $i$'s neighbours that need to be 
influenced for $i$ to be influenced. Intuitively, $\phi_i$ refers to $i$'s willingness to be 
influenced.
For simplicity, we assume every node can be influenced by the same threshold rule, meaning 
for every node $i$ we have $\phi_i = \phi$ for some fixed $\phi$.

\subsection{Influence Networks}
Alongside the rule describing how individuals are able to influence each other, we also need to 
describe who influences whom.
However, social networks are not yet fully understood in terms of their network properties. 
Consequently, we make the following assumptions about our influence network for computer 
simulation.
We assume that every individual $i$ in a population of size $N$ has $n_i$ neighbours, where 
$n_i$ is drawn from an \textit{influence distribution} $p(n)$ with known average $n_{avg}$ that is much less 
than the population size $n_{avg} \ll N$.
Of note is that $n_i$ refers not to the number of individual that $i$ knows but the number of 
other individuals they can influence due different factors such as their character, expertise, 
and community.
Additionally, we explore the scenario where individual influence is unidirectional, meaning 
an individual $i$ can influence its neighbour $j$ but $j$ cannot influence $i$, and 
bidirectional, meaning if $i$ can influence its neighbour $j$ then $j$ can influence $i$. 
This is to represent scenarios where an individual follows other individuals, such as Instagram 
influencers, and influence is spread in one direction and when individuals influence each 
other such as within a family.

To describe the \textit{influence distribution}, we use two common random graph models 
which contain properties similar to real social networks called Scale-Free Networks and 
Poisson Random Graphs.


\subsubsection{Poisson Random Graph}
The Poisson random graph model is a simple random graph model where the influence distribution $p(n)$ is Poisson meaning $p(n) \sim Poisson(\lambda)$ with $\lambda$ representing the average number of neighbours. In a world described by a Poisson random graph, the network exhibits no structure meaning influence connections between neighbours are random. 
Additionally, the influence distribution has little variation around its average meaning individuals who are more influential generally aren't exceptionally more influential.
With this model, connections can be either undirected or have a direction.
Consequently, this influence distribution can model worlds where the spread of influences are bidirectional and unidirectional.
% Consequently, this influence distribution can model worlds where influences are bidirectional, meaning neighbours can influence each other, and worlds where influence is unidirectional, meaning an individual can influence a neighbour but they may not necessarily influence them.

\subsubsection{Scale-Free Networks}
The Scale-Free random graph model is a common model which is used to simulate a network that is more reminiscent of a social network. 
Specifically, the influence distribution $p(n)$ follows a power law distribution meaning $p(n) \sim n^{-\alpha}$ for some $\alpha$.


\subsection{Influentials}
What are influentials and how we define them.
Pagerank.



\subsection{Influence Dynamics}



\begin{itemize}
    \item Threshold Model
    \item Scale-Free network
    \item Poisson Random Graph Model 
    \item Percolation
\end{itemize}


\section{Results}




\section{Conclusion}

\bibliographystyle{unsrt}
\bibliography{bibliography}




\section{Main Points}

\begin{itemize}
    \item Influentials - a minority within a population who exert influence on an exceptional number of their peers.
    \item Motivation - We want to understand whether influentials play a significant role in the spread of influence.
    \item How - What factors within a network contribute to the spread of innovations and does it affect the influence of influentials
    \item 
\end{itemize}


Background 
\begin{itemize}
    \item What are influentials 
    \item 
\end{itemize}

\begin{itemize}
    \item Poisson Random Graph - If a cascade can occur, anyone can start it
\end{itemize}

\section{Introduction}

\subsection{The Influentials Hypothesis}
\texttt{Influentials} are a minority of individuals who influence an exceptional 
number of their peers. 

The \textbf{hypothesis} was that influentials were mediators between the source of innovation and the majority 
of society. The model, called the \texttt{two-step flow} of communication.

\subsection{Questions of Interest}
\begin{itemize}
    \item What does the two-step model say about influentials?
    \item How do influentials exert influence over the larger population?
    \item Are influentials responsible for the spread/diffusion of innovation?
\end{itemize}




\subsection{General Results}

\begin{itemize}
    \item Under certain (rare) conditions, influentials appear responsible for 
    initiating cascades of influence and are important.
    \item Under most conditions, most cascades are driven by easily influenced individuals
    influencing other easily influenced individuals.
\end{itemize}


\begin{itemize}
    \item Model: Two-way influence model with influentials acting as a middle-man.
    \item Simulations suggest that under certain conditions, influentials promote 
    cascading effects but these conditions are rare.
    \item Computer simulation models 
\end{itemize}


Simulations 

Threshold Model
\begin{itemize}
    \item Each individual $i$ in a population of $N$ influence $n_i$ random individuals.
    \item Early adopters are individuals who adopt an innovation when a single neighbour has innovated.
\end{itemize}




\end{document}



